\section{The Core Hypothesis: A Self-Regulating Black Hole Interior}
\label{sec:core_hypothesis}

The standard model of black holes, based on general relativity, predicts that matter within the event horizon undergoes unrestricted collapse, forming a singularity where spacetime curvature and energy density diverge. However, quantum effects, particularly those associated with Hawking radiation, suggest that this picture may be incomplete. We propose a self-regulating mechanism within black holes that prevents singularity formation and leads to an equilibrium core of finite curvature.

\subsection{Postulate 1: Curvature-Dependent Local Radiation}
Hawking radiation is typically derived as a quantum process occurring near the event horizon \cite{hawking1975particle}, where vacuum fluctuations generate particle-antiparticle pairs, with one escaping as radiation. In our framework, we extend this idea by proposing that a similar effect occurs \textit{locally} within the black hole interior, with radiation emission dependent on the local curvature.

Formally, we define the local radiation emission rate as:
\begin{equation}
    \Gamma_H(r) = f(\mathcal{R}(r)) \frac{1}{M^2},
\end{equation}
where:
\begin{itemize}
    \item \( \mathcal{R}(r) \) is the local spacetime curvature at radius \( r \),
    \item \( f(\mathcal{R}) \) is a function encoding the dependence of radiation on curvature,
    \item \( M \) is the black hole mass.
\end{itemize}

We assume that \( f(\mathcal{R}) \) is a monotonic function satisfying:
\begin{equation}
    \frac{df}{d\mathcal{R}} > 0.
\end{equation}
This implies that regions of higher curvature experience stronger radiation effects. 

At the event horizon, where \( \mathcal{R}(r) \approx \mathcal{R}(r_s) \), we recover standard Hawking radiation. However, as we move deeper into the interior, the radiation rate increases due to rising curvature, eventually modifying the internal structure.

\subsection{Postulate 2: Balance Between Radiation and Collapse}
Within the event horizon, classical physics dictates that matter collapses toward \( r=0 \). However, in our model, the increasing radiation flux at high curvature counteracts this collapse. 

We propose that an equilibrium condition is established when the local radiation flux matches the rate of gravitational collapse:
\begin{equation}
    \Gamma_H(\mathcal{R}_c) = \Phi_{\text{collapse}}(\mathcal{R}_c),
\end{equation}
where:
\begin{itemize}
    \item \( \mathcal{R}_c \) is the curvature at equilibrium,
    \item \( \Phi_{\text{collapse}}(\mathcal{R}) \) represents the inward flux due to gravitational collapse.
\end{itemize}

This balance prevents the curvature from diverging to infinity, stabilizing the core at a finite curvature \( \mathcal{R}_c \).

\subsection{Postulate 3: Formation of a Constant Curvature Core}
If equilibrium is achieved, the black hole interior does not collapse into a singularity. Instead, for radii \( r \leq r_c \), the curvature stabilizes at a constant value:
\begin{equation}
    \mathcal{R}(r) = \mathcal{R}_c.
\end{equation}

This implies:
\begin{itemize}
    \item A finite energy density, satisfying Einstein’s equation:
    \begin{equation}
        \mathcal{R}_c \sim 8\pi G \rho_c.
    \end{equation}
    \item A non-singular core that remains dynamically stable.
    \item The possibility that the equilibrium curvature is set by quantum gravity effects, potentially at the Planck scale:
    \begin{equation}
        \mathcal{R}_c \sim \frac{c^3}{\hbar G}.
    \end{equation}
\end{itemize}

\subsection{Physical Interpretation and Implications}
This framework suggests that black holes do not have singularities but instead develop structured interiors. The self-regulating mechanism introduces a feedback loop:
\begin{enumerate}
    \item Higher curvature generates more radiation.
    \item Increased radiation counteracts gravitational collapse.
    \item Equilibrium is reached when the two effects balance.
\end{enumerate}

This model aligns with several non-singular black hole proposals, including regular black holes \cite{bardeen1968non, hayward2006formation}, quantum gravity-based structures \cite{bojowald2005nonsingular, rovelli1996black}, and recent proposals for Planck-scale remnants \cite{brustein2018black}. 

The next section presents the mathematical framework necessary to formalize these equilibrium conditions.
