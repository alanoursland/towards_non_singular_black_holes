% !TeX root = ..\towards_non_singular_black_holes.tex
% -------------------------------------------------
\section{Introduction}
\label{sec:intro}

Black holes lie at the intersection of general relativity (GR) and quantum field theory (QFT): they are the most extreme classical solutions of Einstein's equations yet also the stage on which quantum vacuum fluctuations reveal themselves through Hawking radiation. Despite half a century of progress, the \emph{interior} of a black hole remains poorly understood.  In particular, the singularities predicted by the Hawking-Penrose theorems \cite{penrose1965singularity,hawking1975particle} signal a breakdown of classical physics.

\vspace{4pt}
\paragraph*{The classical picture.}
For Schwarzschild and Kerr spacetimes the energy conditions ensure that infalling matter must reach a spacelike or timelike singularity in finite proper time. No classical mechanism can halt the divergence of curvature once the event horizon has been crossed.

\vspace{4pt}
\paragraph*{Quantum hints.}
Hawking's discovery that black holes radiate thermally \cite{hawking1975particle} established that horizons are not inert. Yet back-reaction calculations in four dimensions are inconclusive: does the interior still develop a singularity as the hole evaporates? Proposals to resolve the issue typically invoke \emph{new UV physics}—string-theoretic fuzzballs, loop-quantum-gravity bounces, Planck stars, or non-singular metrics in asymptotic-safety scenarios \cite{bardeen1968non,hayward2006formation,bojowald2005nonsingular,rovelli1996black,Bonanno2023RegularBH}. By contrast, comparatively little attention has been paid to the possibility that \emph{known} semiclassical effects might themselves arrest collapse.

\vspace{4pt}
\paragraph*{A curvature-triggered feedback.}
In this paper we explore such a semiclassical mechanism. The idea is simple:
\begin{enumerate}[leftmargin=1.1em,label=(\arabic*)]
  \item Below a curvature threshold \(K_{\mathrm{th}}\) the vacuum evolves adiabatically and particle creation is negligible (Parker's criterion \cite{parker1968}).
  \item Once \(K\!>\!K_{\mathrm{th}}\) the adiabatic condition fails and local quantum emission turns on, with a rate \(\Gamma_H\propto K^{3/4}\) motivated by the Unruh temperature \(T_{\mathrm{loc}}=a/2\pi\) \cite{Unruh1976}.
  \item The resulting positive-energy flux propagates outward while a negative-energy partner flows inward, reducing the effective mass density and opposing further collapse.
\end{enumerate}
We show that this feedback generically drives the interior toward a finite-curvature, de Sitter-like core.

\vspace{4pt}
\paragraph*{Why plausibility is not enough.}
To test the idea we need a self-consistent semiclassical model.
We therefore:
\begin{itemize}[leftmargin=*]
  \item derive the emission law and an integral balance condition in Section~\ref{sec:core_hypothesis};
  \item solve the Einstein equations with the one-loop trace-anomaly stress tensor in Section~\ref{sec:backreaction};
  \item analyse linear and preliminary non-linear stability, and discuss observational signatures such as gravitational-wave echoes \cite{CarballoRubio2018,Cardoso2016Echoes} in Section~\ref{sec:stability}.
\end{itemize}

\vspace{4pt}
\paragraph*{Scope and caveats.}
Our analysis is restricted to (i) static, spherically symmetric interiors; (ii) conformal quantum fields at one loop; (iii) a phenomenological match of the emission constant \(C\) to the standard Hawking flux. Rotation, charge, non-local higher-loop effects and full numerical evolution of dynamical collapse are deferred to future work.

\vspace{4pt}
\paragraph*{Significance.}
If borne out, a curvature-triggered feedback would provide a \emph{new semiclassical route} to non-singular black holes, predictive in both its interior geometry and its external signatures. Conversely, observational non-detection of the predicted echoes and shadow shifts would constrain the allowed parameter space, pushing us toward other resolutions of the singularity problem.

The rest of the paper is organised as follows. Section~\ref{sec:core_hypothesis} formulates the interior emission law and energy-balance criterion. Section~\ref{sec:backreaction} derives the constant-curvature core from the semiclassical Einstein equations. Section~\ref{sec:stability} investigates stability and possible observational tests. Section~\ref{sec:conclusions} summarises our findings and outlines extensions needed to integrate the mechanism into a full quantum-gravity framework.
