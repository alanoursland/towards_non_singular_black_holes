\section{Stability and Open Questions}
\label{sec:stability}

The self-regulating black hole core model introduces a structured interior where local Hawking-like radiation counteracts gravitational collapse, leading to an equilibrium curvature at \( \mathcal{R}_c \). However, for this framework to be physically viable, the equilibrium state must be stable against perturbations. Additionally, this model raises fundamental questions regarding black hole thermodynamics, information flow, and long-term evolution.

\subsection{Perturbative Stability of the Core}
A key question is whether small perturbations in curvature or energy density within the core lead to divergence (instability) or are naturally damped (stability). Stability can be analyzed in the following ways:

\subsubsection{Linear Stability Analysis}
We define a small perturbation \( \delta \mathcal{R}(r,t) \) around the equilibrium curvature:
\begin{equation}
    \mathcal{R}(r,t) = \mathcal{R}_c + \delta \mathcal{R}(r,t).
\end{equation}
The evolution of \( \delta \mathcal{R} \) is governed by:
\begin{equation}
    \frac{\partial \delta \mathcal{R}}{\partial t} + v_r \frac{\partial \delta \mathcal{R}}{\partial r} = -\lambda \delta \mathcal{R},
\end{equation}
where:
\begin{itemize}
    \item \( v_r \) represents an effective propagation velocity of curvature perturbations,
    \item \( \lambda \) is a damping (if positive) or growth (if negative) coefficient.
\end{itemize}

If \( \lambda > 0 \), perturbations decay over time, suggesting **stability**. If \( \lambda < 0 \), the core structure is unstable, potentially leading to either collapse or dispersion.

\subsubsection{Energy Density Stability}
Since curvature and energy density are linked by Einstein’s equations, perturbations in \( \rho \) also affect stability. The key question is whether energy density perturbations reinforce equilibrium or cause runaway behavior.

We define:
\begin{equation}
    \rho(r,t) = \rho_c + \delta \rho(r,t),
\end{equation}
where \( \rho_c \) is the equilibrium core density. The governing equation is:
\begin{equation}
    \frac{\partial \delta \rho}{\partial t} + \frac{\partial}{\partial r} \left( \Phi_\text{rad} - \Phi_\text{collapse} \right) = 0.
\end{equation}
If **radiation outflow adjusts dynamically to stabilize \( \rho_c \)**, the core remains stable.

\subsection{Nonlinear Evolution and Late-Stage Dynamics}
A complete treatment of stability must consider **nonlinear effects**. If the radiation response \( f(\mathcal{R}) \) saturates at high curvatures, feedback mechanisms could prevent instability. However, if radiation is not sufficiently responsive, perturbations could grow, leading to:
\begin{itemize}
    \item A gradual deviation from equilibrium, leading to core shrinkage or expansion.
    \item A catastrophic breakdown of the core, leading to renewed collapse or dispersal.
\end{itemize}

A full nonlinear stability analysis requires solving the coupled mass-energy and radiation equations numerically.

\subsection{Information Retention and Black Hole Evolution}
One of the major unresolved problems in black hole physics is the fate of information. If the singularity is avoided in favor of a structured core, several possibilities arise:
\begin{enumerate}
    \item Information could be stored in the core and gradually released during late-stage evaporation.
    \item The core itself could act as a quantum remnant, preventing information loss.
    \item Information may be redistributed internally but remain inaccessible from the exterior.
\end{enumerate}
Determining the core’s role in information flow requires an explicit **quantum description** of the interior radiation process.

\subsection{Late-Stage Evolution and Possible Remnants}
As the black hole loses mass due to Hawking radiation, the equilibrium core should dynamically adjust. Possible end states include:
\begin{itemize}
    \item A shrinking core that eventually evaporates completely.
    \item A stable Planck-scale remnant that persists indefinitely.
\end{itemize}
The remnant hypothesis is appealing because it could provide a mechanism for resolving the black hole information paradox, but further work is required to determine whether a self-regulating core naturally transitions into a remnant.

\subsection{Open Questions}
While this framework provides a compelling alternative to classical singularity formation, several questions remain:
\begin{enumerate}
    \item \textbf{Derivation of the Radiation Function:} Can \( f(\mathcal{R}) \) be derived from first principles in quantum field theory or quantum gravity?
    \item \textbf{Stability Under Perturbations:} Do numerical simulations confirm that the equilibrium core remains stable under small fluctuations?
    \item \textbf{Modified Einstein Equations:} How should the field equations be modified to include quantum backreaction effects?
    \item \textbf{Observational Signatures:} Could deviations from classical predictions be detected via gravitational wave echoes or other indirect observations?
    \item \textbf{Final State of Black Holes:} Does this model favor total evaporation or remnant formation?
\end{enumerate}

The next section will summarize our findings and outline future directions for further investigation.
